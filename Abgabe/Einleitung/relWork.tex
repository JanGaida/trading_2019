\section{Related Work}
Im Folgenden werden weitere Projekte aufgezeigt, die den Gedanken eines handelsfähigen Bots
erfolgreich umgesetzt haben. Diese weisen sowohl Gemeinsamkeiten als auch Unterschiede zum hier
vorgestellten Konzept auf, welche anhand geeigneter Programme vorgestellt werden.\newline Ein solcher
Trading Bot wird mit der Open Source Lösung „Gekko“ umgesetzt.
Hierbei handelt es sich um eine kostenlose Open Source Trading Plattform, die mit
Kryptowährungen, in diesem Fall Bitcoins arbeitet.\newline Im Gegensatz zu der von der Studiengruppe
angestrebten Lösung wird hier also nur eine Währung realisiert. Außerdem ermöglicht sie es aktuelle,
aber auch vergangene Kurse genau zu verfolgen den Zwecken entsprechend zu analysieren.\newline Gekko
erlaubt unteranderem sogar sogenannte „Backtests“. Das sind Strategiesimulationen, die anhand von
historischen Daten aufzeigen, ob bestimmte Trades in der Vergangenheit profitabel gewesen wären.
Ebenfalls interessant ist, dass man neben auswählbaren, auch eigene Strategien übergeben kann.\newline
Aber auch mit Echtzeitdate können Strategien verfolgt und automatisch Trades bei entsprechenden
Signalen gesetzt setzen. Anders als in diesem Projekt wurde die Lösung in Javascript implementiert
und läuft auf der Plattform Node.js, vgl. \cite{gekko2}.
\newline Ein weiteres Bitcoin-Trading Model stellt der Haasbot dar. Der Bot ist zwar für Bitcoin konzipiert,
bietet aber dafür hohe sonstige Auswahl- und Einstellungsmöglichkeiten. Dieser Anbieter gibt den
Nutzern bereits zu Beginn die Möglichkeit zwischen mehreren Basis-Bots wählen und weitere
hinzuzufügen. Auch eigene Script-Bots können erstellt werden. Analysen, Sicherheiten und
Absicherungen gehören zu dem Konzept.Haasbot unterstützt zudem eine Vielfalt an
unterschiedlichen Börsen, darunter auch bereits vorgestellte Projekt Bitfinex.\newline Dieses Modell
unterscheidet außerdem nicht nur zwischen unterschiedlichen Trading Strategien, sondern auch
zwischen unterschiedlichen Bot-Typen. Dazu gehören Arbitrage-, Order-, Script und Haasbot Trading
Bots. Im Gegensatz zu Gecko, sind die für Haasbot notwendigen Lizenzen aber auch recht kostspielig, vgl. \cite{haasbot}.\newline
Eine andere Strategie verfolgt der HodlBot. Dieser basiert auf dem Konzept des "passiven
Investierens".\newline Hierbei wird im Gegensatz zu den meisten andern Bots, nicht auf einzelne Währungen
gesetzt, sondern bestimmte Marktindizes, die sogenannten „Hodl Indices“. Kunden erstellen hierfür
zuerst ein persönliches Portfolio. In dieses halten sie fest, welche Währungen sie bevorzugen und
welches Gewichtungsschema bzw. welche Strategie verfolgt werden soll. Mittels verschiedener APIs
werden anschließend automatisch Trades abgeschlossen, vgl. \cite{gekko}.\newline
Letztendlich gibt es neben unserer Arbeit also bereits eine Vielzahl an Trading Bots, welche jeweils
Vorteile und Nachteile aufweisen. An diesen wird sich anhand dieses Projektes orientiert. Die
Recherche zeigt auch, dass Trading Bots von unterschiedlichsten Herstellern weitestgehend, den hier
aufgezeigten Grundprinzipien folgen.
