\section{Datenbezug über die API}

Gemäß des in der Einleitung beschriebenen Data Warehouse Gedanken müssen zunächst geeignete Datenquellen herangezogenen werden, bevor mit den daher bezogenen Informationen weiter verfahren werden kann. In dem Fall dieses Projekt müssen daher geeignete Daten eingebunden werden, auf welcher Basis der entwickelte Bot arbeitet. Hierfür wurden die historischen Daten für eine größere Menge an Kryptowährungen gesammelt und dann aufbereitet. Die Projektgruppe arbeitete schlussendlich mit 416 verschiedenen Kryptowährungen. Die historischen Daten hierfür wurden mittels der von der Börse für Kryptowährungen bereitgestellten Schnittstelle bezogen, vgl. \cite{bitfinex}. Um von dieser effizient für die eigenen Zwecke Gebrauch zu machen wurden sich an der entsprechenden Dokumentation und allgemein zu findende Tutorials orientiert, vgl. \cite{kaggle}.\newline Hierbei wurden für ein jedes Währungspapier die historischen Daten innerhalb einer bestimmten Zeitperiode geladen und gespeichert. Es wurde ein beliebiges Startdatum gewählt. Als Enddatum wird der Tag vor demjenigen Tag gewählt an welchem die Funktion zuletzt ausgeführt wird. Für diesen Zeitraum wird täglich der jeweils letzte Wert („End of day“) abgefragt. Nachdem die Daten für die entsprechenden Daten gesammelt wurden, wurden sie zur Interpolation weitergereicht.